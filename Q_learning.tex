\section{Q-learning}
\label{sec:q_learning}

An alternative approach for learning to select good actions (the so-called `control problem') when we don't know the transition structure of the world is to learn the value of taking action $a$ in state $s$ instead of only learning the value of the state.
This gives rise to `Q-learning' methods, where we learn the \emph{state-action values}
\begin{equation}
    \label{eq:Q-values}
    Q_\pi(s,a) =  \mathbb{E}_{a \sim \pi} \left [ \sum_{t'=t} \gamma^{t' - t} r_{t'} | s_{t} = s, a_{t} = a \right ].
\end{equation}
Given these state-action values, it is simple to perform action selection by taking the action with the highest expected value in a given state,
\begin{equation}
    a^*(s) = \text{argmax}_{a} Q(s, a).
\end{equation}
To learn the $Q$-values necessary for such action selection, we start by expanding \Cref{eq:Q-values},
\begin{equation}
    \label{eq:Q-expanded}
    Q_\pi(s_t,a_t) =  r(s_t, a_t) + \gamma \sum_{s_{t+1}} p(s_{t+1} | s_t, a_t) \sum_{a_{t+1}} \pi(a_{t+1} | s_{t+1}) Q_\pi(s_{t+1}, a_{t+1}).
\end{equation}
For the optimal policy $\pi^*(s) := \delta(a, a^*(s))$, this gives rise to a self-consistency expression for the optimal state-action values:
\begin{equation}
    \label{eq:Q-optimal}
    Q_{\pi^*}(s_t,a_t) =  r(s_t, a_t) + \gamma \mathbb{E}_{s_{t+1} \sim p(s_{t+1} | s_t, a_t)} \left [ \text{max}_{a_{t+1}} Q_{\pi^*}(s_{t+1}, a_{t+1}) \right ].
\end{equation}
Importantly, this self-consistency expression only holds when the Q-values have converged to the the true expected rewards, and the associated greedy policy is therefore optimal \citep{sutton2018reinforcement}.
We can therefore use \Cref{eq:Q-optimal} as a loss function by defining
\begin{equation}
    \mathcal{L} \propto \left (  Q(s_t,a_t) - (r(s_t, a_t) + \gamma \mathbb{E}_{s_{t+1} \sim p(s_{t+1} | s_t, a_t)} \left [ \text{max}_{a_{t+1}} Q(s_{t+1}, a_{t+1}) \right ] ) \right )^2,
\end{equation}
Gradient descent w.r.t $Q(s_t,a_t)$ gives us an update rule
\begin{align}
    \Delta  Q(s_t,a_t) & \propto - \frac{\partial \mathcal{L}}{\partial  Q(s_t,a_t)}\\
    &\propto - Q(s_t,a_t) + r(s_t, a_t) + \gamma \mathbb{E}_{s_{t+1} \sim p(s_{t+1} | s_t, a_t)} \left [ \text{max}_{a_{t+1}} Q(s_{t+1}, a_{t+1}) \right ] \\
    &\approx - Q(s_t,a_t) + r_t + \gamma \text{max}_{a_{t+1}} Q(s_{t+1}, a_{t+1}).
\end{align}
This is the so-called Q-learning update rule \citep{watkins1989learning}, where we have again estimated the expectation over next states with the single sample actually seen by the agent.

Q-learning is guaranteed to converge to the true Q-values in the limit of infinitesimal learning rates and infinite sampling of the state-action space \citep{watkins1992q,sutton2018reinforcement}.
However, following the greedy policy $a(s) = \text{argmax}_{a} Q(s, a)$ before convergence of the Q values can lead to undersampling of the state-action space and poor performance.
It is therefore common to either use an `$\epsilon$-greedy' policy, $\pi(a|s) = \epsilon / |\mathcal{A}| + (1-\epsilon) \delta(a, \text{argmax}_{a} Q(s, a))$, or a softmax-policy, $\pi(a|s) \propto \text{exp}(\beta Q(a, s))$, to collect the experience used to update the Q values.
These approaches give rise to so-called `off-policy' algorithms, since the policy used in the learning update (the greedy policy) is different from the policy used for action selection (the stochastic policy).
An on-policy version of Q-learning known as `SARSA' (state-action-reward-state-action) is also commonly used, where the Q-learning update uses the Q-value of the sampled action instead of the greedy action:
\begin{align}
    \Delta  Q(s_t,a_t) \propto - Q(s_t,a_t) + r_t + \gamma Q(s_{t+1}, a_{t+1}).
\end{align}
This will converge to the true Q values for a given policy $\pi$, similar to how the TD learning rule in \Cref{eq:TD-learning} is guaranteed to converge to the true value function for a given policy, again under assumptions of infinite sampling of the space.
Interestingly, studies in the neuroscience literature have also found evidence for midbrain dopamine neurons encoding something like the prediction error used for Q learning \citep{roesch2007dopamine} or SARSA \citep{morris2006midbrain} in `instrumental conditioning' settings, where animals have to choose between multiple actions with different values \citep{niv2009reinforcement}.

