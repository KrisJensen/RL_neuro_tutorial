\begin{figure*}[!t]
    \centering
    \vspace*{-0.5em}
    \includegraphics[width=0.95\textwidth]{./figs/Q.pdf}
    \caption[Q]{\label{fig:Q}
        %
        {\bfseries Q-learning.}
        {\bfseries (A)}~Empirical reward as a function of episode number for Q-learners with different levels of stochasticity in their policy ($\epsilon \in \{0, 0.1, 0.2\}$; legend).
        For these simulations, we used a learning rate of $\delta = 0.05$ for all agents and no temporal discount factor ($\gamma = 1$).
        The agent with $\epsilon = 0$ converges to an optimal policy, similar to the TD agent in \Cref{fig:TD}A.
        However, convergence is in this case slower despite using the same learning rate, because the Q-learner has to learn about each action independently while the TD agent used its one-step world model to aggregate learning across actions going to the same state.
        In this cliffworld environment, increasing epsilon leads to worse performance since it increases the probability of falling of the cliff.
        Additionally, there is no risk of getting stuck in a local minimum since there is only one rewarding state, which decreases the value of exploration.
        {\bfseries (B)}~As in (A), now for a non-cliffworld grid environment with two goals: one with a reward of +20 at location (0, 4), and one with a reward of +50 at location (5,0).
        In this case, having non-zero epsilon can increase the probability of discovering the `high reward' goal rather than getting stuck with a locally optimal policy of moving to the `low reward' goal.
        In these simulations, we used a learning rate of $\delta = 1$, since this effect is less robust with lower learning rates that lead to more exploration of the environment across all agents.
        {\bfseries (C)}~Policy learned by a Q-learning and SARSA agent with $\epsilon = 0.2$.
        Colours indicate the maximum value of any action in a state from blue (low) to yellow (high), and arrows indicate which action has the highest value.
        The Q-learning agent learns to move right above the cliff because this is the optimal thing to do assuming that subsequent actions are also optimal.
        This is because it is an `off-policy' algorithm that does not take into account the actual policy of the agent.
        In contrast, the SARSA agent learns to move a `safe distance' away from the cliff since it is an `on-policy' algorithm that takes into account the finite probability of the agent choosing to move off the cliff from any state along the edge.
        In many cases, a Q-learning agent is also trained using a stochastic $\epsilon$-greedy policy but evaluated with the greedy policy corresponding to $\epsilon = 0$.
        }
    \vspace*{-1.0em}
\end{figure*}

