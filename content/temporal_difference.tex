\section{Temporal difference learning}
\label{sec:temporal_difference}

A simple way to mazimize reward in an environment is to learn the `value' of different states, and then move towards states with high value.
The potential importance of such an algorithm for neuroscience is evident from the value-seeking behaviour of many organisms, and the widespread findings of neural `codes' for value across the brain \citep{rushworth2011frontal}.
This leaves the question of how such value codes can be learned in a biologically plausible setting.

A simple answer to this question takes the form of the classical `temporal difference learning' algorithm \citep{sutton1988learning,sutton2018reinforcement}.
This involves defining a \emph{value function} for a given state $s$ and policy $\pi$, which specifies the expected future reward when following $\pi$ starting from $s$:
\begin{equation}
    \label{eq:V-values}
    V_{\pi}(s) = \mathbb{E}_{\tau \sim p_\pi(\tau)} \left [ \sum_{t' \geq t} \gamma^{t' - t} r_{t'} | s_t = s \right ].
\end{equation}
Here, $\mathbb{E}_{\tau \sim p_\pi(\tau)} [ \cdot ]$ indicates an expectation taken over trajectories $\tau$ resulting from the agent following the policy $\pi$.
For the true value function, we can expand this as a self-consistency equation
\begin{align}
    V_{\pi}(s) &= r(s) + \sum_{s'} p_{\pi}(s_{t+1} = s' | s_t = s) \mathbb{E}_{\tau \sim p_\pi(\tau)} \left [ \sum_{t' = t+1} \gamma^{t' - t} r_{t'} | s_{t+1} = s' \right ] \\
    &=  r(s) + \gamma \sum_{s'} p_{\pi}(s_{t+1} = s' | s_t = s) V_{\pi}(s'),
    \label{eq:value_expansion}
\end{align}
where $p_{\pi}(s_{t+1} = s' | s_t = s) = \sum_a \pi(a|s) p(s_{t+1} = s' | s_t = s, a_t = a)$ is the probability of transitioning from $s$ to $s'$ under $\pi$.
Importantly, \Cref{eq:value_expansion} would not hold if $V_{\pi}(s)$ was not the true value function \citep{sutton2018reinforcement}.
%
When learning an approximate value function $V(s)$, we can therefore use this bootstrapped self-consistency relation as an objective function:
\begin{equation}
    \mathcal{L} \propto \left (  V(s) - \left ( r(s) + \gamma \mathbb{E}_{p_\pi(s'|s)} \left [ V(s') \right ] \right ) \right )^2,
\end{equation}
Gradient descent w.r.t $V(s)$ gives us an update rule
\begin{align}
    \Delta V(s) &\propto - \frac{\partial \mathcal{L}}{\partial V(s)}\\
    \label{eq:TD-learning_exp}
    &\propto - V(s) + r(s) + \gamma \mathbb{E}_{p_\pi(s'|s)} \left [ V(s') \right ]\\
    \label{eq:TD-learning}
    &\approx - V(s_t) + r_t + \gamma V(s_{t+1}).
\end{align}
The last line approximates the expected update with a single sample corresponding to the states and reward actually experienced.
As more and more experience is collected and many small gradient steps are taken according \Cref{eq:TD-learning}, these single-sample estimates will average out to the expectation in \Cref{eq:TD-learning_exp} (\Cref{fig:TD}A).
Variants of this algorithm can also learn about multiple past states at once using the notion of \emph{eligibility traces} \citep{sutton2018reinforcement}.
However, \Cref{eq:TD-learning} is the canonical temporal difference learning rule \citep{sutton1988learning}, and it leads to learning dynamics where the temporal difference error $\delta := - V(s_t) + r(s_t) + \gamma \mathbb{E}_{p_\pi(s_{t+1})} \left [ V(s_{t+1}) \right ]$ is progressively propagated from the rewarding state to prior states that predict the upcoming reward.

This gradual propagation of value prediction errors from rewarding states to prior states that predict upcoming reward has been of great interest in neuroscience.
In particular, classical work by \citet{schultz1997neural} demonstrated a similar pattern of neural activity in dopaminergic VTA neurons, which formed the foundation of a now well-established theory that dopamine provides a biological prediction error signal that drives reward-guided learning \citep{niv2009reinforcement}.
At the behavioural level, such theories of dopamine as a reinforcing signal are consistent with experiments showing that stimulation of dopamine neurons can be a strong driver of learning in instrumental conditioning tasks, where animals have to take explicit actions to achieve rewarding outcomes \citep{olds1954positive, tsai2009phasic}.
%A wealth of other research also suggests a close correspondence between midbrain dopaminergic neural activity and temporal difference learning, and we refer to \citet{niv2009reinforcement} and others for a more comprehensive overview.

\begin{figure*}[!t]
    \centering
    \vspace*{-0.5em}
    \includegraphics[width=0.95\textwidth]{./figs/TD.pdf}
    \caption[TD]{\label{fig:TD}
        %
        {\bfseries Temporal difference learning.}
        {\bfseries (A)}~Value functions aquired through temporal difference learning (\Cref{eq:TD-learning}) while acting according to either a random (top) or an optimal (bottom) policy.
        These simulations were performed with a random start state in the cliffworld environment to ensure full coverage of the space.
        Dark blue indicates negative expected reward (-100) and yellow indicates positive expected reward (+50).
        For these simulations, we used a learning rate of $\alpha = 0.05$ and no temporal discounting ($\gamma = 1$).
        Under the random policy, states near the cliff have low value even though they are close to the goal, since the agent will often fall off the cliff from there.
        Under the optimal policy, all states have high expected reward, since the agent will always find the goal.
        States nearer the goal have higher value than those further away, although this is hard to distinguish on this color scale.
        {\bfseries (B)}~Empirical reward as a function of episode number for a TD-learning agent that acts according to \Cref{eq:value_action_selection} while updating its value estimates according to \Cref{eq:TD-learning}.
        For this agent, action selection assumes access to a `one-step' world model in order to evaluate the consequence of each putative action.
        We see that the agent gradually converges to an optimal policy.
        Parameters for the agent are as in (A).
        {\bfseries (C)}~Value function learned by a greedy TD agent as in (B), plotted either early (top) or late (bottom) in training.
        Early in training, the agent has learned that the cliff is bad but doesn't know where the goal is or how to get there.
        Late in training, the agent has learned a value function that locally resembles the optimal value function from (A), while it has not learned the value of distant states that are rarely or never visited.
        This is a potential shortcoming of `greedy' agents that can easily converge to a sub-optimal local maximum in more complicated environments.
        For this analysis, we used a high learning rate of $\alpha = 0.5$ to make the early TD updates larger and therefore more visible.
        }
    \vspace*{-1.0em}
\end{figure*}



Much classical work in neuroscience has focused on value learning in Pavlovian conditioning tasks, but animals in natural environments also have to take actions on the basis of this information.
However, having learned a value function, it is simple to use it for action selection if we can estimate $p(s' | s, a)$.
In this case, the expected reward associated with action $a$ can be written as
\begin{equation}
    \mathbb{E} \left [\sum_{t' = t} \gamma^{t'-t} r_{t'} | s_t = s, a_t = a \right ] = r(s, a) + \gamma \sum_{s'} p( s' | s, a) V(s').
\end{equation}
We can then perform action selection by choosing the action with the highest expected reward,
\begin{equation}
    \label{eq:value_action_selection}
    a^*(s) = \text{argmax}_{a} \left ( r(s, a) + \gamma \sum_{s'} p(s' | s, a) V(s') \right ).
\end{equation}
Updating the value function according to \Cref{{eq:TD-learning}} while acting in the environment according to \Cref{{eq:value_action_selection}} leads to an agent that gradually learns to take better actions as it learns a better value function (\Cref{fig:TD}B-C).
This provides a potentially biologically plausible algorithm for reward-driven learning in agents with access to a one-step predictive model.

