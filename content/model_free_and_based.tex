\section{Model-free and model-based reinforcement learning}

In the previous section, we developed a so-called `model-free' reinforcement learning algorithm.
This involves learning a stimulus-response pattern that says `when in state $s$, take action $a$'.
Such algorithms do not require much computation at decision time, where they rely on cached state or action values.
However, it can require a lot of experience with the environment to learn these model-free policies, and they can be inflexible in changing environments.
This is incompatible with many aspect of animal behaviour, which we know is adaptive and can benefit from `latent learning' in an environment before a reward-driven task is encountered in the first place \citep{blodgett1929effect,tolman1948cognitive}.

On the other hand, `model-based' reinforcement learning uses a model of the world to simulate the consequences of different actions at decision time.
This can be much more data efficient, since learning a world model is often simpler than learning a full policy (\Cref{fig:MB}A).
First learning a model of the environment, and then using it to guide reward-driven behaviour also provides one plausible explanation for latent learning and other types of rapid adaptation.
However, model-based decision making can be computationally intensive at decision-time, which is a challenge for animals that rely on rapid decision making for survival (\Cref{fig:MB}B).

In model-based RL, an approximate transition-and-reward function $\tilde{p}(s_{t+1}, r_t | s_t, a_t)$ is learned from past experience.
Once this model has been learned, it can be used for planning at decision time.
This can be done for example by expanding the Q-value relation from \Cref{eq:Q-optimal}:
{\small
\begin{align}
    \label{eq:Q-search}
    Q(s_t,a_t) &\approx  r_t + \gamma \mathbb{E}_{\tilde{p}(s_{t+1} | s_t, a_t)} \left [ \text{argmax}_{a_{t+1}} Q(s_{t+1}, a_{t+1}) \right ]\\
    &\approx r_t + \gamma \mathbb{E}_{\tilde{p}(s_{t+1} | s_t, a_t)} \left [ \text{argmax}_{a_{t+1}}
    \left [ r(s_{t+1}, a_{t+1}) + \gamma \mathbb{E}_{\tilde{p}(s_{t+2} | s_{t+1}, a_{t+1})} \left [ \text{argmax}_{a_{t+2}} Q(s_{t+2}, a_{t+2}) \right ] \right ] \right ] \\
    &= \ldots
\end{align}
}
If the environment is determinstic, $p(s_{t+1} | s_t, a_t)$ is a delta function, and otherwise the next-state expectations may need to be approximated with multiple samples.
Unfortunately, the optimization over all possible action sequences in \Cref{eq:Q-search} is in general an exponentially large search problem in the planning depth, which makes it infeasible for any reasonably sized problem setting.
It is therefore common to either use `depth-first search' with limited breadth, or `breadth-first search' with limited depth.
In breadth-first search, we consider all possible actions at each level of the search tree but terminate the search at a finite depth, instead using cached `model-free' state-values to estimate the reward-to-go from the termination of the search.
This kind of `plan-until-habit' algorithm has also been proposed as a model of human behaviour \citep{keramati2016adaptive}.
In depth-first search, we instead sample a series of paths from $s_t$ all the way to termination (or some upper bound), using a heuristic to prioritize actions expected to be good, and then pick a path associated with high expected reward \citep{huys2012bonsai}.

\begin{figure*}[!t]
    \centering
    \vspace*{-0.5em}
    \includegraphics[width=0.95\textwidth]{./figs/MB_rev.pdf}
    \caption[MB]{\label{fig:MB}
        %
        {\bfseries Model-based reinforcement learning.}
        {\bfseries (A)}~Learning curves for model-based (MB) and model-free (MF) RL agents.
        The MB agent used a depth-first search to compute an optimal path at each decision point, gradually learning the reward and transition functions from exploring the environment.
        The MF agent was a simple Q-learning agent with $\epsilon = 0$ and learning rate $\alpha = 1$.
        {\bfseries (B)}~Wallclock time needed to run 100 episodes of cliffworld with either the MB or MF agents from (A), as a function of the length of the environment.
        While the MB agent required less experience to learn a good policy, the wallclock time per episode was much larger than for the MF agent.
        This illustrates an important balance between model-based and model-free reinforcement learning, where MF methods usually require more experience but MB methods require more compute at decision time.
        {\bfseries (C)}~Learning curve for an agent using the successor representation (SR) together with learning curves for the model-based agent in (A) and the greedy TD-agent from \Cref{fig:TD}.
        The goal was moved from location (9, 0) to location (0, 4) at episode 40 (vertical black line), and location (9, 0) was instead given a reward of -5.
        The MB and SR agents had their reward functions updated to reflect this change and rapidly adapted their policies, while the TD agent had no such mechanism for robustness to changing reward functions.
        Reward curves were convolved with a Gaussian kernel ($\sigma = 3$ episodes), which is why performance appears to decrease slightly before episode 40 for the TD agent.
        The TD and SR agents were assumed to have access to a 1-step world model at initialization, while the MB agent learned the transition structure from experience.
        This is why the MB agent does not exhibit faster initial learning than the SR and TD agents.
        The MB agent also exhibits a small drop in performance around the change in reward because the agents were greedy and did not all learn the full transition function.
        {\bfseries (D)}~An SR agent cannot always adapt to a new reward function if the newly rewarded states have low probablity under the old policy.
        Value functions are shown for an agent that learned an initial policy in an environment with a small reward in the upper left state and intermediate reward in the upper right state (left column).
        A new large reward was then introduced in either the top left state (top row) or bottom left state (bottom row).
        In the second case, the agent was unable to adapt since the old policy had low probability of reaching the large reward, even after initially going to the left.
        This results in a low expected value for going left from the start state (red circles).
        {\bfseries (E)}~Learning curve for a standard Q-learning agent (blue) or DYNA agents that perform different numbers of Q value updates for each step of physical action (legend).
        The DYNA updates simply used cached experience rather than data from a learned world model.
        DYNA agents are able to make better use of limited experience at the cost of increased compute (proportional to the number of updates).
        }
    \vspace*{-1.0em}
\end{figure*}



A shortcoming of these approaches from a neuroscientific perspective is that they often perform a large number of node expansions at each decision point.
While this can be done very quickly on a computer, it seems unrealistic as a model of human decision making.
In some settings, it is possible to use more efficient algorithms once a model has been learned, such as Djikstra's algorithm or $A^*$-search in shortest-path problems \citep{hart1968formal}.
However, it is in general necessary to somehow trade off the temporal opportunity cost of planning with the improvement in policy and expected reward \citep{botvinick2014computational,agrawal2022temporal}.
This has been a popular research area in cognitive science, where a wealth of literature on `resource-rational' decision making has emerged in recent years \citep{griffiths2019doing,callaway2022rational}.
However, this literature has often considered optimal behaviour in simple tasks with less focus on the learning process and neural mechanisms that might implement the necessary computations.
Bridging this gap, recent work has suggested that frontal cortex might initially store a `model-free' policy in its network state, which is gradually updated with model-based information from the hippocampal formation until the policy improvement is outweighed by the temporal opportunity cost of planning \citep{jensen2023recurrent}.
%In this work, the authors found substantial policy improvements from only a few rollouts \citep{vul2014one}, and the `reaction times' of the agent exhibited similar task-related structure to human behaviour.

While several model-based and model-free reinforcement learning methods have thus been developed and used to model animal learning and behaviour, it remains an open question when and whether these different strategies drive animal behaviour.
A different line of research has explicitly investigated the trade-offs between model-based and model-free RL in biological agents \citep{daw2005uncertainty, geerts2020general, lengyel2007hippocampal}, where the choice between the two approaches is thought to be guided by some notion of optimality on the basis of available resources and uncertainty about the environment.
A popular paradigm for such studies attempting to distinguish between model-free and model-based behaviour is the so-called `two-step' task developed by Daw and colleagues (\citealp{daw2011model,momennejad2017successor,wang2018prefrontal}; although note \citealp{akam2015simple}).

It has generally been accepted that humans can use both model-free and model-based decision making, with the dorsolateral striatum being particularly important for model-free reinforcement learning \citep{yin2004lesions, yin2005role}, and the dorsomedial striatum, prefrontal cortex, and hippocampal formation being important for model-based decision making \citep{vikbladh2019hippocampal,geerts2020general,miller2017dorsal,niv2009reinforcement,killcross2003coordination}.
This also has interesting parallels to recent work in the motor learning literature, where the basal ganglia were found to be sufficient for `habitual' motor sequences even in the absence of motor cortex, while motor cortex was necessary for more flexible motor behaviours that are likely to require a high-level `schema' of the task structure \citep{mizes2023motor,mizes2023dissociating}.
In \Cref{sec:scalar_rew} we will see how combining these model-based and model-free ideas with deep learning can also lead to human-level performance in tasks such as chess and Go that require long-term planning, and how such algorithms can provide inspiration for future studies of neuroscience and behaviour.
