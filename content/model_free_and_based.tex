\section{Model-free and model-based reinforcement learning}

In the previous section, we developed a so-called `model-free' reinforcement learning algorithm.
This involves learning a stimulus-response pattern that says `when in state $s$, take action $a$'.
Such algorithms do not require much computation at decision time, where there is no need to plan far into the future.
However, it can require a lot of experience with the environment to learn these model-free policies, and they can be very inflexible in changing environments.

On the other hand, `model-based' reinforcement learning uses a model of the world to simulate the consequences of different actions at decision time.
This can be much more data efficient, since learning a world model is often simpler than learning a full policy (\Cref{fig:MB}A).
However, model-based decision making can also be computationally intensive at decision-time in complicated state spaces (\Cref{fig:MB}B).
Such models have recently exhibited impressive performance across a range of RL settings with large state spaces, including atari, chess, shogi, and Go \citep{silver2018general, schrittwieser2020mastering, deisenroth2011pilco}.

In model-based RL, experience is used to learn a transition-and-reward function $p(s_{t+1}, r_t | s_t, a_t)$ from past experience.
Once this model has been learned, it can be used for planning at decision time.
This can be done for example by expanding the Q-value relation from \Cref{eq:Q-optimal}:
{\small
\begin{align}
    \label{eq:Q-search}
    Q(s_t,a_t) &=  r_t + \gamma \mathbb{E}_{p(s_{t+1} | s_t, a_t)} \left [ \text{argmax}_{a_{t+1}} Q(s_{t+1}, a_{t+1}) \right ]\\
    &= r_t + \gamma \mathbb{E}_{p(s_{t+1} | s_t, a_t)} \left [ \text{argmax}_{a_{t+1}}
    \left [ r(a_{t+1}, s_{t+1}) + \gamma \mathbb{E}_{p(s_{t+2} | s_{t+1}, a_{t+1})} \left [ \text{argmax}_{a_{t+2}} Q(s_{t+2}, a_{t+2}) \right ] \right ] \right ] \\
    %&= r_t + \sum_{s_{t+1}} p(s_{t+1} | s_t, a_t) \text{argmax}_{a_{t+1}}
    %\left [ r(a_{t+1}, s_{t+1}) + \sum_{s_{t+2}} p(s_{t+2} | s_{t+1}, a_{t+1}) \text{argmax}_{a_{t+2}}
    %\left [ r(a_{t+2}, s_{t+2}) + \sum_{s_{t+3}} p(s_{t+3} | s_{t+2}, a_{t+2}) \text{argmax}_{a_{t+3}} Q(s_{t+3}, a_{t+3}) \right ] \right ]
    &= \ldots
\end{align}
}
If the environment is determinstic, $p(s_{t+1} | s_t, a_t)$ is a delta function, and otherwise the next-state expectations may need to be approximated with multiple samples.
Unfortunately, the optimization over all possible action sequences in \Cref{eq:Q-search} is in general an exponentially large search problem in the planning depth, which makes it infeasible for any reasonably sized problem setting.
It is therefore common to either use `depth-first search' with limited breadth, or `breadth-first search' with limited depth.
In breadth-first search, we consider all possible actions at each level of the search tree but terminate the search at a finite depth, instead using cached `model-free' state-values to estimate the reward-to-go from the termination of the search.
This kind of `plan-until-habit' algorithm has also been proposed as a model of human behavior \citep{keramati2016adaptive}.
In depth-first search, we instead sample a series of paths from $s_t$ all the way to termination (or some upper bound), using a heuristic to prioritize actions expected to be good, and then pick a path associated with high expected reward.
This is what is implemented in most Monte-Carlo tree search (MCTS) methods, such as AlphaZero \citep{silver2018general}, which uses a hardcoded transition function, and MuZero \citep{schrittwieser2020mastering}, which uses a learned transition function.
In both of these approaches, neural networks are used (i) to estimate the `reward-to-go' if the maximum search depth is reached before termination, and (ii) to select which actions to evaluate during the search process.
Methods for learning such value functions and policies in neural networks are covered in more detail in \Cref{sec:deep_RL}.

\begin{figure*}[!t]
    \centering
    \vspace*{-0.5em}
    \includegraphics[width=0.80\textwidth]{./figs/MB.pdf}
    \caption[MB]{\label{fig:MB}
        %
        {\bfseries Model-based reinforcement learning.}
        {\bfseries (A)}~Learning curves for model-based (MB) and model-free (MF) RL agents.
        In this case, the MB agent used a depth-first search to compute an optimal path at each decision point.
        If a shorter path to a given state had already been discovered, the search was not continued.
        The reward and transition functions of the agent were initialized to zero and the empty set respectively, and they were gradually populated as the agent explored the environment across episodes.
        The MF agent was a simple Q-learning agent with $\epsilon = 0$ and learning rate $\alpha = 1$.
        The MB agent requires substantially less experience than the MF agent for a given level of performance.
        {\bfseries (B)}~Wallclock time needed to run 100 episodes of cliffworld with either the MB or MF agents from (A), as a function of the length of the environment.
        Although (A) shows that the MB agent requires substantially less experience to learn a good policy, the wallclock time per episode was much larger than that required by the MF agent.
        This illustrates an important balance between model-based and model-free reinforcement learning, where MF methods usually require more experience but MB methods require more compute at decision time.
        In domains such as robotics, where collecting data is very time consuming, MB methods can still be a more efficient way to reach a high level of performance.
        {\bfseries (C)}~Learning curve for an agent using the successor representation (SR) together with learning curves for the model-based agent in (A) and the greedy TD-agent from \Cref{fig:TD}.
        At episode 40 (vertical black line), the goal was moved from location (9, 0) to location (0, 4), and location (9, 0) was instead given a reward of -5.
        The MB and SR agents had their reward functions updated to reflect this change and were immediable able to adapt their policies, while the TD agent had no such mechanism for robustness to changing reward functions.
        It is worth noting that an SR agent cannot always adapt to a new reward function if the newly rewarded states have low probablity under the old policy, since the successor matrix itself is a function of the policy.
        Reward curves were convolved with a Gaussian kernel ($\sigma = 3$ episodes), which is why performance appears to decrease slightly before episode 40 for the TD agent.
        In this simulation, the TD and SR agents were assumed to have access to a 1-step world model at initialization, while the MB agent had to learn the transition structure from experience.
        This is why the MB agent does not exhibit faster initial learning than the SR and TD agents.
        {\bfseries (D)}~Learning curve for a standard Q-learning agent (blue) or DYNA agents that perform different numbers of Q value updates for each step of physical action (legend).
        In this case, the DYNA updates simply used cached experience rather than data from a learned world model.
        We see that DYNA agents are able to make better use of limited experience, although this comes at the cost of increased compute (proportional to the number of updates).
        }
    \vspace*{-1.0em}
\end{figure*}



A shortcoming of these MCTS-based approaches from a neuroscientific point of view is that they often perform very many node expansions.
For example, MuZero and AlphaZero perform 800 rollouts to a leaf node for each decision that they make.
While this can be done very quickly on a computer, it seems unrealistic as a model of human decision making.
In some settings, it is possible to use more efficient algorithms once a model has been learned, such as Djikstra's algorithm or $A^*$-search in shortest-path problems \citep{hart1968formal}.
However, in general it is necessary to somehow trade off the temporal opportunity cost of searching with the improvement in policy and expected reward \citep{botvinick2014computational}.
This has been a popular research area in cognitive science, where a wealth of literature on `resource-rational' decision making has emerged in recent years \citep{griffiths2019doing,callaway2022rational}.
However, this literature has often considered optimal behavior in simple tasks with less focus on the learning process and neural mechanisms that might implement the necessary computations.
Bridging this gap, recent work has suggested that frontal cortex might initially store a `model-free' policy in its network state and gradually update this with model-based information from the hippocampal formation in cases where the policy improvement is expected to make up for the temporal opportunity cost \citep{jensen2023recurrent}.
In this work, the authors found substantial policy improvements from only a few rollouts \citep{vul2014one}, and the `reaction times' of the agent exhibited similar task-related structure to human behaviour in a simple navigation task.

While we thus have a range of both model-based and model-free reinforcement learning methods in the machine learning literature, it has been an open question which ones best describe the behaviour and computations of humans and other animals.
A wealth of research has investigated the trade-offs between these two approaches and when one might be used over the other \citep{daw2005uncertainty, geerts2020general, lengyel2007hippocampal}, which is usually thought to be guided by some notion of optimality on the basis of available resources and uncertainty about the environment.
A popular paradigm for distinguishing between model-free and model-based behaviour is the so-called `two-step' task of Daw and colleagues (\citealp{daw2011model,momennejad2017successor,wang2018prefrontal}; although note \citealp{akam2015simple}).
It has generally been accepted that humans can use both model-free and model-based decision making, with the dorsolateral striatum being particularly important for model-free reinforcement learning \citep{yin2004lesions, yin2005role}, and the dorsomedial striatum, prefrontal cortex, and hippocampal formation being important for model-based decision making \citep{vikbladh2019hippocampal,geerts2020general,miller2017dorsal,niv2009reinforcement,killcross2003coordination}.
This has interesting parallels to recent work in the motor learning literature, where the basal ganglia were found to be sufficient for `habitual' motor sequences even in the absence of motor cortex, while motor cortex was necessary for more flexible motor behaviours that are likely to require a high-level `schema' of the task structure \citep{mizes2023motor,mizes2023dissociating}.


