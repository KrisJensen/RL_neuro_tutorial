\section{Deep reinforcement learning}
\label{sec:deep_RL}

So far, we have considered the setting of small state- and action-spaces, for which we can simply learn a tabular policy.
However, in many realistic settings, the state space is large enough that we cannot enumerate all possible states and actions.
In these cases, we instead have to rely on \emph{function approximation} \citep{sutton2018reinforcement}.
This approach makes the assumption that similar states will also be associated with similar state-action value functions and should therefore have similar policies.
By making this assumption, we can generalize to unseen states based on our previous experience, provided that our function approximator is sufficiently good.
Recent years have seen incredible progress in this setting by using deep or recurrent neural networks as powerful function approximators for reinforcement learning -- the domain of `deep reinforcement learning'.
This approach has seen an increase in interest not just in the machine learning literature, but recently also as a model of neural dynamics and behaviour in humans and other animals \citep{wang2018prefrontal, jensen2023recurrent, makino2023arithmetic, merel2019deep, banino2018vector}.
For a more comprehensive overview of the links between deep RL and neuroscience, we refer the interested reader to \citet{botvinick2020deep}.
Modern approaches in (model-free) deep RL can largely be divided into two categories: policy gradient methods and deep Q learning.
In policy gradient methods, we directly train the network to output a policy, whereas deep Q networks (DQNs) output a set of state-action values that can be used for action selection.

\subsubsection*{Policy gradient methods}

The conceptually simplest approach for deep reinforcement learning relies on the so-called policy gradient methods \citep{sutton2018reinforcement}.
In this approach, we train a neural network with parameters $\theta$ to take as input the (observable) state of the environment and directly output a policy $\pi_\theta$.
Our goal is then to find the setting of $\theta$ that maximizes the expected reward in \Cref{eq:RL_objective}.
A conceptually simple way to achieve this would be to define $R_\tau := \sum_{t=0}^T \gamma^t r_t$ and use gradient descent with gradients given by
\begin{align}
    \nabla_\theta J(\theta) &= \nabla_\theta \mathbb{E}_{\tau \sim \pi_\theta} \left [ R_\tau \right ]\\
    &= \sum_{\tau} R_\tau \nabla_\theta p_\theta(\tau).
    \label{eq:simple_deriv}
\end{align}
Here, $\tau \sim \pi_\theta$ indicates trajectories sampled from the distribution over trajectories induced by policy $\pi_\theta$, and $J(\theta)$ indicates the expectation of $R_\tau$ under $\pi_\theta$ (c.f. \Cref{eq:RL_objective}).
However, evaluating \Cref{eq:simple_deriv} requires us to know how the environment will respond to our actions in order to compute the necessary derivatives, which in general may not be the case (although some model-based deep RL approaches train a differentiable predictive model of the environment to do this; \citealp{clavera2020model}).
Instead, we use the `log-derivative trick', which takes advantage of the linearity of the expectation and the identity $\nabla_\theta \log f(\theta) = f(\theta)^{-1} \nabla_\theta f(\theta)$ to write
\begin{align}
    \label{eq:deriv_J}
    \nabla_\theta J(\theta) & = \sum_\tau R_\tau \nabla_\theta p_\theta(\tau) \\
                            & = \sum_\tau R_\tau p_\theta(\tau) \nabla_\theta \log p_\theta(\tau) \\
                            & = \mathbb{E}_{\tau \sim \pi_\theta} \left [ R_\tau \nabla_\theta \log p_\theta(\tau) \right ],
\end{align}
Since the environment does not depend on $\theta$, we can simplify the calculation of $\nabla_\theta \log p_\theta (\tau)$:
\begin{align}
    \label{eq:deriv_log_ptau}
    \nabla_\theta \log p_\theta(\tau) & = \nabla_\theta \left [ \log p(s_0) + \sum_{t=0}^T \log p(s_{t+1} | s_t, a_t) + \log \pi_\theta (a_t|s_t) \right ] \\
                                      & = \sum_{t=0}^T \nabla_\theta \log \pi_\theta (a_t|s_t).
\end{align}
Inserting \Cref{eq:deriv_log_ptau} in \Cref{eq:deriv_J} and taking a Monte Carlo estimate of the expectation, we arrive at the REINFORCE algorithm \citep{williams1992simple}:
\begin{align}
    \nabla_\theta J(\theta) & = \mathbb{E}_{\tau \sim \pi_\theta} \left [ R_\tau \sum_{t=0}^T \nabla_\theta \log \pi_\theta (a_t|s_t) \right ]                                         \\
                            & \approx \frac{1}{N} \sum_{\tau \sim \pi_\theta} \left ( \sum_{t=0}^T \gamma^t r_t \right ) \left ( \sum_{t=0}^T \nabla_\theta \log \pi_\theta (a_t|s_t) \right ).
    \label{eq:orig_reinforce}
\end{align}
Intuitively, \Cref{eq:orig_reinforce} says that we should preferentially upregulate the probability of trajectories with high reward, which will lead to downregulation of trajectories with lower reward due to the normalization of the policy.
Importantly, \Cref{eq:orig_reinforce} no longer requires us to differentiate through the environment -- only the policy.

While the REINFORCE algorithm is unbiased, it also has high variance, which can make learning slow and unstable.
It is therefore common to introduce modifications, which can help reduce the variance.
The first of these comes from noting that an action taken at time $t$ cannot affect the reward received at times $t'<t$.
This allows us to define $R_t := \sum_{t'=t}^T \gamma^{t'-t} r_{t'}$ and rewrite our REINFORCE update as
\begin{equation}
    \label{eq:reinforce}
    \nabla_\theta J(\theta) \approx \frac{1}{N} \sum_{\tau \sim \pi_\theta}  \sum_{t=0}^T R_t \nabla_\theta \log \pi_\theta (a_t|s_t).
\end{equation}
This is the formulation most commonly used, but it is worth noting that it is actually \emph{not} the same as \Cref{eq:orig_reinforce}, which would require us to use $R_t= \sum_{t'=t}^T \gamma^{t'-0} r_{t'}$.
As briefly discussed in \Cref{sec:problem_setting}, this is because the discount factor $\gamma$ is generally used as a variance reduction method rather than an indication that we intrinsically care less about rewards later in the task, although \Cref{eq:reinforce} is also consistent with an interpretation of $(1-\gamma)$ as a termination probability.

It is also straightforward to show that for a baseline $B(s_t)$ that does not depend on $a_t$,
\begin{align}
    \mathbb{E}_{s_t, a_t \sim p_\theta(s_t, a_t)} \left [ B(s_t) \nabla_\theta \log \pi_\theta (a_t|s_t) \right ]
     & = \int_{s_t} B(s_t) p(s_t) \left [ \nabla_\theta \int_{a_t} \pi_\theta (a_t|s_t) da_t \right ] ds_t \\
     & = \int_{s_t} B(s_t) p(s_t) \left [ \nabla_\theta 1 \right ] ds_t = 0.
\end{align}
A corollary of this result is that we can subtract such a baseline from our empirical reward and still have an unbiased estimator while potentially reducing its variance.
A common choice here is the expected future reward
\begin{equation}
    V(s_t) = \mathbb{E} \left [ \sum_{t' = t}^T \gamma^{t'=t} r_{t'} | s_t \right ]
\end{equation}
This gives rise to the so-called `actor-critic' algorithm
\begin{equation}
    \label{eq:AC}
    \nabla_\theta J(\theta) \approx \frac{1}{N} \sum_{\tau \sim \pi_\theta}  \sum_{t=0}^T (R_t - V(s_t)) \nabla_\theta \log \pi_\theta (a_t|s_t).
\end{equation}
Intuitively, \Cref{eq:AC} says that we should upregulate the probability of actions that lead to higher-than-expected reward and downregulate the probability of actions that lead to lower-than-expected reward.
Interestingly, many studies in the neuroscience literature have suggested that the brain could be using something akin to an actor-critic algorithm, with dorsal striatum implementing the `actor' and ventral striatum the `critic' \citep{takahashi2008silencing,sutton2018reinforcement}.

Finally, it is common to reduce the variance of our gradient estimator through an approach known as `bootstrapping'.
To implement this, we note that
\begin{equation}
    \mathbb{E} \left [ R_t \right ] = \mathbb{E}[r_t + \gamma V(s_{t+1})].
\end{equation}
This is useful because $r_t + \gamma V(s_{t+1})$ has lower variance than $R_t$, so using this bootstrapped estimate of the reward-to-go can reduce the noise in our gradient estimate when using finite sample sizes.
We therefore use $r_t + \gamma V(s_{t+1})$ to define the `advantage function' $A_t = r_t + \gamma V(s_{t+1}) - V(s_t)$, which gives rise to the `advantage actor-critic' (A2C) algorithm:
\begin{equation}
    \nabla_\theta J(\theta) \approx \frac{1}{N} \sum_{\tau \sim \pi_\theta}  \sum_{t=0}^T (r_t + \gamma V(s_{t+1}) - V(s_t)) \nabla_\theta \log \pi_\theta (a_t|s_t).
\end{equation}
While we have considered two extreme cases of a full Monte Carlo estimate of $R_t$ and a `one-step' bootstrap, the sum in $R_t$ can be truncated to any order with $R_{t'}$ replaced by $V(s_{t'})$.
In theory, this estimator remains unbiased if our value function is correct.
However, in practice the estimate of $V(s_{t'})$ learned by our critic will be inexact, which introduces a bias to our parameter updates.
The bootstrapping procedure outlined above thus leads to a tradeoff between the bias and variance of our parameter updates, and the optimal tradeoff will depend on the problem setting, network architecture, and training paradigm being used.

For these actor-critic algorithms, it is common to parameterize both the policy $\pi_\theta(a_t|s_t)$ and the value function $V_\phi(s_t)$ with neural networks -- often with a subset of shared parameters between $\theta$ and $\phi$.
This leaves the problem of updating the parameters.
In order to use out-of-the-box automatic differentiation software, we do this by defining an auxiliary utility (i.e. negative loss)
\begin{equation}
    \label{eq:Jtilde}
    \tilde{J}(\theta) = \frac{1}{N} \sum_{\tau \sim \pi_\theta}  \sum_{t=0}^T (R_t - V(s_t)) \log \pi_\theta (a_t|s_t),
\end{equation}
where $R_t$ and $V(s_t)$ are treated as constant w.r.t. $\theta$.
In \Cref{eq:Jtilde} and the following equations, we can use either the empirical reward-to-go $R_t = \sum_{t' \geq t} \gamma^{t'-t} r_{t'}$, or we can use the bootstrapped estimate $R_t \approx r_t + \gamma V(s_{t+1})$.
While $\tilde{J}(\theta)$ has no intrinsic interpretation, it is chosen such that
\begin{equation}
    \nabla_\theta \tilde{J}(\theta) = \nabla_\theta J(\theta).
\end{equation}

The gradient of the value function loss (i.e. negative utility) is given by
\begin{equation}
    \nabla_\phi \mathcal{L}_V = \nabla_\phi \sum_t \frac12  (R_t - V_\phi(s_t))^2 = \sum_t - (R_t - V_\phi(s_t)) \nabla_\phi V_\phi(s_t).
\end{equation}
When $\pi_\theta$ and $V_\phi$ share parameters, we also need to balance the gradients w.r.t. $\theta$ and $\phi$ in our update step.
We do this by defining $\delta_t := R_t - V(s_t)$ (or $\delta_t := r_t + \gamma V(s_{t+1}) - V(s_t)$), introducing a hyperparameter $\beta_V$, and maximizing the joint utility
\begin{equation}
    \mathcal{L} = \frac{1}{N} \sum_{\tau \sim \pi_\theta} \sum_{t=0}^T ( \log \pi_\theta (a_t|s_t) + \beta_V V_\phi(s_t) ) \delta_t .
\end{equation}
Importantly, we do not propagate gradients w.r.t. $\phi$ or $\theta$ through the computation of $\delta_t$.
Finally, it is also common to add an auxiliary entropy loss of the form $\mathcal{L}_E = \beta_E \frac{1}{N} \sum_{\tau \sim \pi_\theta} \sum_{t=0}^T \sum_a \pi_\theta (a_t|s_t) \log \pi_\theta (a_t|s_t)$ to encourage exploration.

\begin{figure*}[!t]
    \centering
    \vspace*{-0.5em}
    \includegraphics[width=0.95\textwidth]{./figs/metaRL.pdf}
    \caption[metaRL]{\label{fig:meta}
        {\bfseries Meta-reinforcement learning.}
        The results in this figure reproduce some of the preliminary analyses in Figure 1 of \citet{wang2018prefrontal}.
        {\bfseries (A)}~We trained a recurrent meta-reinforcement learning agent in a two-armed bandit task, where the reward probabilities of each arm were sampled independently from $\mathcal{U}(0, 1)$ at the beginning of each episode and remained fixed throughout the episode.
        A recurrent neural network was trained across many episodes with \emph{different reward probabilities} using an actor-critic algorithm.
        The input to the agent consisted of the previous action, the previous reward, and the time-within-trial.
        The average reward per episode is plotted against the episode number, showing that the agent gradually learns to adapt within each episode to the particular instantiation of the bandit task.
        Importantly, the parameters of the network are fixed within an episode, meaning that this adaptation occurs through the recurrent dynamics.
        Dashed horizontal lines indicate the reward of an agent selecting random actions and an `oracle' agent that always chooses the best arm.
        {\bfseries (B)}~Heatmap showing example behaviour of the agent in episodes with different reward probabilities for the first arm, $p(r | a = 1)$.
        For these plots, the probability of reward from the second arm was set to $p(r | a = 2) = 1 - p(r | a = 1)$.
        Across episodes, the agent experiments with different actions and eventually converges on taking the optimal action.
        For episodes with more symmetric reward probabilities (near the middle), it takes longer for the agent to identify the optimal action.
        This balance between exploration and exploitation is mediated by the recurrent network dynamics, which are learned over many episodes using deep reinforcement learning.
        {\bfseries (C)}~We averaged the hidden state of the RNN over 100 episodes for each of several different reward probabilities, ranging from high reward for action 1 (blue) to low reward for action 1 (green)
        We then performed PCA on the resulting matrix of average hidden states to compute a low-dimensional hidden state trajectory over the course of an episode for each reward probability.
        This figure shows the 2-dimensional embedding of neural activity in the meta-RL agent, which converges to different regions of state space throughout the episode for different reward probabilities.
        Black cross indicates the hidden state at the beginning of an episode, and coloured points indicate the final hidden state in an episode for the different reward probabilities.
        }
    \vspace*{-1.0em}
\end{figure*}



While these policy gradient methods may seem far removed from neuroscience, they have had several applications in the field in recent years.
For example, \citet{li2022integrating} recently demonstrated that an off-policy variant known as the `soft actor-critic' algorithm \citep{haarnoja2018soft} can be used to train an agent that shapes the behaviour of \textit{C. elegans} via an optogenetic action space.
Networks trained with policy gradient algorithms have also recently attracted significant interest as models of learning and neural dynamics in the biological brain \citep{wang2018prefrontal, jensen2023recurrent, merel2019deep}.
Of particular interest, \citet{wang2018prefrontal} suggested that frontal cortex can be well described as a \emph{recurrent} deep RL agent, where the RNN parameters are configured by learning from rewards over long periods of time from many tasks that have a shared underlying structure.
Importantly, this `slow' model-free learning process gives rise to an agent that can rapidly learn from experience with \emph{fixed parameters} when exposed to a new task from the same distribution.
This is achieved by the agent learning to effectively implement a fast RL-like algorithm in the \emph{dynamics} of the network (\Cref{fig:meta}).
This process, whereby an agent trained slowly on a large distribution of tasks can rapidly adapt to a new task, is known as `meta-reinforcement learning' and is a popular area of research in machine learning \citep{finn2017model, ritter2018been, duan2016rl, wang2016learning}.
\citet{wang2018prefrontal} showed that such as meta-RL model can explain a range of neuroscientific findings, including
\begin{itemize}
    \item Dynamic adaptation of the effective learning rate of an agent to the volatility of the environment \citep{behrens2007learning}.
    \item The emergence of `model-based' behaviour in the `two-step' task commonly used to distinguish between model-free and model-based RL \citep{miller2017dorsal,daw2011model}.
    \item The ability of animals to get progressively faster at learning when exposed to multiple tasks with a consistent abstract task structure \citep{harlow1949formation}.
\end{itemize}
Recently, \citet{jensen2023recurrent} also extended the work of \citet{wang2018prefrontal} to allow the meta-RL agent to learn not just from physical experience, but also from imagined experience using a learned model of the environment.
This resulted in an agent that computes an intial model-free policy, which can be adaptively improved with model-based computations when the resulting policy improvement makes up for the temporal opportunity cost of the model-based computation.
Interestingly, the response times of the resulting agent mirrored those of human participants in a maze navigation task, where the agent similarly spent more time `thinking' early in a trial and far from the goal.

\subsubsection*{Deep Q-learning}

While the policy gradient methods discussed above are useful, they also have limitations.
One is that most policy gradient methods are fundamentally `on-policy', which means that the data used for learning must be sampled from the agent itself.
This is necessary for the log derivative trick to work.
Another is that the gradient can have fairly high variance despite all of our variance reduction efforts.
We can begin to tackle both of these shortcomings using a method known as `deep Q-learning'.
To do so, we first note that our actor-critic algorithm above required us to estimate the value function $V_\phi(s_t)$.
However, as we saw in \Cref{eq:value_action_selection}, such value functions can directly be used for action selection without having to fit a separate policy network.
We also noted previously that this requires simulating the results of the actions, which can be inconvenient.
As in the tabular setting, we can circumvent this problem by directly fitting state-action values (`Q-values'), now using function approximation.
This gives rise to the family of `deep Q-learning' methods, which closely mirror the tabular Q-learning considered previously, but now with function approximation.

The general recipe involves defining a Q function $Q_\theta(a, s)$, where the parameters $\theta$ of the deep network defining our agent are learned as follows:
\begin{itemize}
    \item Collect experience $(s_t, a_t, r_t, s_{t+1})$.
    \item Define a loss $\mathcal{L} = 0.5 [ Q_\theta(s_t, a_t) - (r_t + \gamma \text{max}_a Q_\theta(s_{t+1}, a)) ]^2 $.
    \item Update the network parameters $\Delta \theta \propto - \frac{\partial \mathcal{L}}{\partial \theta}$.
\end{itemize}
The above steps can either be run `online' using the experience of the agent as it is being generated, or it could occur `in batch', whereby a collection of experiences is first generated, followed by optimization of the Q network.
The `max' in the target value ensures that we continuously improve our policy.
When acting according to our policy, we simply pick the action predicted to have the highest value, usually using some variant of $\epsilon$-greedy or softmax to introduce stochasticity and variability into our trajectories.

On the surface, the above looks like a straightforward generalization of tabular Q-learning, and it may seem surprising that deep Q-learning did not see significant use or success until the foundational work of \citet{mnih2013playing}.
However, a major difficulty arises from the autocorrelation of the states observed by the agent, which leads to the network `chasing' a target function $(r_t + \gamma \text{max}_a Q_\theta(s_{t+1}, a))$ that changes with the environment.
A critical advance that overcame the resulting instability was the use of an experience replay buffer, whereby the experience generated by the agent is added to a global replay buffer $\mathcal{B}$.
One or more \emph{random} experiences are then sampled from the buffer and used to update the network parameters -- reminiscent of the `DYNA' architecture described previously.
An additional algorithmic instability arises from the fact that $\mathcal{L}$ includes the term $\text{max}_a Q_\theta(s_{t+1}, a)$, which we cannot take gradients through despite its dependence on $\theta$.
This means that we are trying to optimize our parameters for a continuously moving target, which destabilitizes the optimization process.
To combat this, it is common to use a `target network' $Q_{\theta'}(s_{t+1}, a)$ that is fixed for a period of time, usually after being set to a copy of our `student network'.
Together, these two approaches give rise to the `deep Q network' (DQN) developed by \citet{mnih2013playing}, which is trained as follows:
\begin{itemize}
    \item Collect experience $(s_t, a_t, r_t, s_{t+1})$ and add to $\mathcal{B}$ [optionally many iterations].
    \item Sample a \emph{new} experience $(s'_t, a'_t, r'_t, s'_{t+1}) \sim \mathcal{B}$ [optionally a full batch].
    \item Define a loss $\mathcal{L} = 0.5 [ Q_\theta(s'_t, a'_t) - (r'_t + \gamma \text{max}_a Q_{\theta'}(s'_{t+1}, a)) ]^2 $ [optionally averaged over the full batch]. Note that we are training the student network with parameters $\theta$ while using a target network with parameters $\theta'$ inside the max.
    \item Update the network parameters $\Delta \theta \propto - \frac{\partial \mathcal{L}}{\partial \theta}$.
    \item Once we have repeated the above a sufficient number of times, set our target network to the student network, $\theta' \leftarrow \theta$.
\end{itemize}
As indicated above, this procedure usually involves averaging over a full batch of experience when computing gradients.
Experience can also be collected in batch, and usually `stale' experiences are periodically removed from $\mathcal{B}$.
It is worth noting that this algorithm is effectively off-policy, since most experience in $\mathcal{B}$ was collected by a policy defined by an old set of parameters $\theta$ -- and the data in $\mathcal{B}$ can in fact be generated completely independently of the agent we are training.
Finally, even though the above is more stable than naive deep Q-learning, an additional instability arises from the fact that $Q_{max}(s'_{t+1}) = \text{max}_a Q_{\theta'}(s'_{t+1}, a)$ uses the same Q values both to estimate which action is best and what the value of that action is, which leads to a positively biased estimate.
This can be mitigated by using \emph{separate} Q networks to select the best action and evaluate its value, $Q_{max}(s'_{t+1}) \leftarrow Q_{\theta'}(s'_{t+1}, \text{argmax}_a(Q_{\theta}(s'_{t+1}, a)))$, in an approach known as `double Q-learning' \citep{van2016deep}.
Commonly, this is done by simply using the student network for action selection and the target network to evaluate its expected value.

While DQNs have shown impressive performance across a range of machine learning settings \citep{mnih2013playing, lillicrap2015continuous, schaul2015prioritized, kalashnikov2018qt}, they have seen less interest from the neuroscience community.
This is perhaps paradoxical given the prevalence of tabular Q-learning in the theoretical neuroscience literature.
An interesting exception to this is recent work by \citet{makino2023arithmetic}, which shows parallels between the values learned by a DQN and neural representations in mammalian cortex during a compositional behavioural task.
Additionally, the importance of experience replay in DQNs \citep{mnih2013playing, schaul2015prioritized} has close parallels to the proposal that hippocampal replay constitutes a form of experience replay \citep{mattar2018prioritized} -- although the latter was only analyzed in a tabular setting.

