
\section{Learning from scalar rewards -- or not?}
\label{sec:scalar_rew}

The methods discussed so far have all relied on some scalar `reward' available from the environment.
Distributional RL predicted the full distribution of possible rewards, but it still assumed that an external reward was eventually observed.
This is perhaps reasonable in many experimentally controlled settings, where we deliver juice or water to animals as a function of their actions.
It is also appropriate in machine learning settings, where we want to optimize some externally imposed objective that can be quantified.
However, in natural environments, there is no such natural definition of `reward' -- instead, different actions can be rewarded in different ways and must be converted to an internal learning signal \citep{juechems2019does}.
For example, foraging for food might reduce hunger, while collecting water might reduce thirst.
Starting a family might have an initial energy cost, but it could improve survival later in life and propagate the gene pool.
An RL purist would have to convert all of these different gains and losses into a single common unit and balance them appropriately to decide what actions to take \citep{silver2021reward}.
However, it has also been suggested that biological networks adaptively change the instantaneous objective of an agent via a higher order controller that determines what the current lower-level objective is \citep{juechems2019does,miller2001integrative}.
This resembles ideas in hierarchical reinforcement learning \citep{pateria2021hierarchical}, but where the higher-order policies are not necessarily learned through reinforcement learning within a lifetime, instead emerging during evolution where survival depended on optimizing `useful' objectives.

A related challenge is the technical difficulty of learning from scalar rewards, which are generally sparse.
While the methods in \Cref{sec:deep_RL} can train a neural network to maximize reward in theory, in practice they are often noisy, unstable, or find local minima.
As a topical example, consider the challenge of training large language models to interact with human users.
Training such a model from scratch using reinforcement learning would be near-impossible, but `pretraining' the model on a large unsupervised dataset followed by `reinforcement learning from human feedback' has proven hugely successful and revolutionized large language models for human interactions \citep{team2023gemini}.
This works because the large unlabelled dataset provides a way to learn good \emph{representations} that distinguish different concepts, since this is necessary to solve the base task of predicting the next word.
Once such representations are learned, the subsequent finetuning for human interactions is an easier problem that can be solved with reinforcement learning.

Of more relevance to neuroscience, it is also common in deep RL to introduce additional `auxiliary costs' to the utility function that are jointly optimized together with reward maximization, and which can use a richer data source to drive representation learning in the network \citep{jaderberg2016reinforcement}.
A popular approach is to include predictive losses that require the agent to predict the next observation from the current state and action \citep{jaderberg2016reinforcement, zintgraf2019varibad}.
Such predictive losses have close parallels in neuroscience, where it has been suggested that predictive learning could result in many of the representations observed in biological circuits \citep{rao1999predictive, stachenfeld2017hippocampus, whittington2020tolman, blanco2021dopamine} and serve as a foundation for model-based planning \citep{jensen2023recurrent}.
\citet{fang2023predictive} also recently showed that augmenting an RL agent with an auxiliary predictive objective leads to neural representations that resemble biological circuits more closely.
This suggests a potentially important interaction between self-supervised representation learning and reward-driven reinforcement learning in biological circuits.

If a good predictive model of the world is learned, this model can also be used to turn the reinforcement learning problem into a supervised learning problem, which will generally have lower variance gradients.
Consider for example the seminal MuZero algorithm developed by \citet{schrittwieser2020mastering}, which was used to train a network to solve Atari, chess, shogi, or Go.
This neural network was trained to predict future representations and reward, and then used Monte Carlo tree search (MCTS; a model based-search algorithm c.f. \Cref{sec:model_free_based}) to select good actions.
These actions, which had been optimized using MCTS, were then treated as a supervised learning target by training the base policy network to predict the actions selected by the MCTS algorithm.
Through many iterations of such predictive learning, MCTS-based action selection, and policy improvement by gradient descent, the network learned both the transition structure of the task and how to select good actions.
%Importantly, the MCTS-based policy improvement step relied on learning a good world model for simulating the effects of different actions.
Such semi-supervised approaches to reinforcement learning are useful in settings where learning the transition structure of the world is easier than learning a policy \citep{jensen2023recurrent}, which is often the case.

Finally, algorithms that seek to maximize scalar rewards often run into challenges related to exploration.
In particular, once an above-average policy has been identified, exploration is disfavored because it has lower expected reward, even if there is a finite probability of discovering new and better policies.
This is why imposing stochasticity in the form of e.g. $\epsilon$-greedy policies is common in the tabular RL literature.
In deep RL, it is instead popular to include various forms of `exploration bonuses' in the objective function.
Policy gradient algorithms for example often add an auxiliary entropy loss of the form $\mathcal{L}_E = \sum_{\tau \sim p_{\pi_\theta}(\tau)} \sum_{t=0}^T \sum_a \pi_\theta (a_t|s_t) \log \pi_\theta (a_t|s_t)$, and indeed the simple example in \Cref{fig:meta} does not work without such an entropy loss.
Other approaches to improve exploration include hierarchical reinforcement learning, which introduces autocorrelations in the policy to improve coverage of the state space.
This is reminiscent of the biological `Lévy flight' hypothesis, which suggests that animals explore an environment using a heavy-tailed distribution of `step sizes' to maximize the probability of finding sparse rewards \citep{viswanathan1999optimizing}.
Biological agents are also thought to engage in periods of `random exploration' that can be interpreted as a biological parallel to $\epsilon$-greedy-like algorithms.
Random exploration appears to be under noradrenergic control \citep{tervo2014behavioral, dubois2021human}, while the zona incerta can drive more directed exploration \citep{ahmadlou2021cell}.
Such neural control of exploration is also well-established in the zebra finch song circuit \citep{olveczky2005vocal}, and it is consistent with a view of top-down imposition of flexible reward functions -- where one possible objective could be to increase variability or reduce uncertainty -- rather than optimization of a global scalar reward.



