\section{Problem setting}
\label{sec:problem_setting}

Here we provide a short introduction to the reinforcement learning problem in a discrete state and action space with a finite time horizon.
For a more general treatment, we refer to \citet{sutton2018reinforcement}.
In the discrete problem setting, the environment consists of states $s \in \mathcal{S}$, and the agent can take actions $a \in \mathcal{A}$.
The environment is characterized by transition and reward probabilities $p(s_{t+1}, r_t | s_t, a_t)$, where $r_t$ is the reward at time $t$.
We will write $r_t = r(s_t, a_t)$ to denote either the reward when it depends deterministically on the state and action, or its expectation otherwise.
We will further make the \emph{Markov assumption} that the next state only depends on the current state and action, $p(s_{t+1}, r_t | s_t, a_t, s_{t-1}, a_{t-1}, ..., s_0, a_0) = p(s_{t+1}, r_t | s_t, a_t)$.

\begin{figure*}[!t]
    \centering
    \vspace*{-0.5em}
    \includegraphics[width=0.85\textwidth]{./figs/schematics.pdf}
    \caption[RL schematics]{\label{fig:schematics}
        %
        {\bfseries The reinforcement learning problem and cliffworld environment.}
        {\bfseries (A)}~Illustration of the reinforcement learning problem.
        An agent (the chick) has to interact with the world to maximize its lifetime reward.
        This involves a balance between exploring potentially interesting states (e.g. learning to fly) while also exploiting states known to yield high reward (e.g. sitting in the nest and eating food brought back by its parents).
        At any given point in time, the chick is in some state $s_t$ from which it can take an action $a_t$, with the probability of different actions determined by the `policy' $\pi(a|s_t)$, which is controlled by the agent.
        $a_t$ then leads to a change in the environment according the non-controllable environment dynamics $s_{t+1}, a_t \sim p(s_{t+1}, r_t | s_t, a_t)$.
        Here, $r_t$ is the empirical `reward' received by the agent, and its objective is to collect as much cumulative reward as possible.
        Often, reinforcement learning problems are divided into `episodes', with the agent learning over the course of multiple repeated exposures to the environment.
        This could for example consist of the chick learning over the course of multiple days when to wake up in anticipation of its parents bringing back food.
        {\bfseries (B)}~The `cliffworld' environment, which will be used to demonstrate the performance and behaviour of a range of reinforcement learning algorithms in this work.
        The agent starts in the lower left corner (location [0, 0]), and the episode finishes when it encounters either the `cliff' (dark blue) or the goal (yellow; location [9,0]).
        If the agent walks off the cliff, it receives a reward of -100.
        If it finds the goal, it receives a reward of +50.
        In any other state, it receives a reward of -1.
        Such negative rewards for `neutral' actions are commonly used to encourage the agent to achieve its goal as fast as possible.
        The arrows indicate the `optimal' policy, which takes the agent to the goal via the shortest possible route that avoids the cliffs.
        }
    \vspace*{-1.0em}
\end{figure*}



We can now define a \emph{trajectory} $\tau = \{ s_t, a_t, r_t \}_{t = 0}^T$, where
\begin{equation}
    p(\tau) = p(s_0) \prod_{t = 0}^T p(s_{t+1}, r_t | s_t, a_t) p(a_t|s_t).
\end{equation}
$p(a_t|s_t)$ is the probability of taking action $a_t$ in state $s_t$, which is usually controlled by the agent and denoted a \emph{policy} $\pi(a_t|s_t)$ (\Cref{fig:schematics}A).
The objective of the agent is to maximize the expected total discounted reward
\begin{align}
    \label{eq:RL_objective}
    J(\pi) &= \mathbb{E}_{\tau \sim p(\tau)} \left [ R_\tau \right ] = \mathbb{E}_{\tau \sim p(\tau)} \left [ \sum_{t=0}^T \gamma^t r_t | \tau \right ],
\end{align}
where $R_\tau := \sum_{t=0}^T \gamma^t r_t | \tau$ and we have written $J(\pi)$ since the policy uniquely specifies $J$ in a stationary environment.
In \Cref{eq:RL_objective}, $\gamma$ is a `discount factor', which stipulates that we should care more about immediate rewards than rewards far in the future.
We can provide three interpretations for this discount factor.
One is that people intrinsically care more about immediate reward than distant reward.
A second is that there is a finite probability $(1-\gamma)$ of the current `episode' or environment terminating or changing at each timestep, in which case we should weight putative future reward by the probability that we are still engaged in the task at that time.
The third view is that $\gamma$ simply provides a tool for reducing the variance of our learning methods, especially in temporally extended tasks.
This third view is most compatible with the fact that \emph{evaluation} of our RL agents at the end of training is generally done without any discounting.

Since $J(\pi$) depends on the policy of the agent, it is possible to search in the space of policies for one that maximizes $J$, which is the topic of reinforcement learning.
It is often assumed that the experience $\tau$ is generated by the agent acting according to its policy, and the resulting experience is then used to update the policy in a way that increases $J(\pi)$ (although note the existence of `off-policy' and `offline' RL; \citealp{levine2020offline}; \Cref{sec:off-policy}).



