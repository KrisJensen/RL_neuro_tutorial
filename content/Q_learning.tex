\section{Q-learning}
\label{sec:q_learning}

\begin{figure*}[!t]
    \centering
    \vspace*{-0.5em}
    \includegraphics[width=0.95\textwidth]{./figs/Q.pdf}
    \caption[Q]{\label{fig:Q}
        %
        {\bfseries Q-learning.}
        {\bfseries (A)}~Empirical reward as a function of episode number for Q-learners with different levels of stochasticity in their policy ($\epsilon \in \{0, 0.1, 0.2\}$; legend).
        For these simulations, we used a learning rate of $\delta = 0.05$ for all agents and no temporal discounting ($\gamma = 1$).
        The agent with $\epsilon = 0$ converges to an optimal policy, similar to the TD agent in \Cref{fig:TD}A.
        However, convergence is in this case slower despite using the same learning rate, because the Q-learner has to learn about each action independently, while the TD agent used its one-step world model to aggregate learning across actions reaching the same state.
        In this cliffworld environment, increasing epsilon leads to worse performance since it increases the probability of falling of the cliff.
        Additionally, there is no risk of getting stuck in a local minimum since there is only one rewarding state, which decreases the value of exploration.
        Lines and shading indicate mean and standard error across 10 simulations.
        {\bfseries (B)}~As in (A), now for a non-cliffworld grid environment with two goals: one with a reward of +20 at location (0, 4), and one with a reward of +50 at location (5,0).
        In this case, having non-zero epsilon can increase the probability of discovering the `high reward' goal rather than getting stuck with a locally optimal policy of moving to the `low reward' goal.
        In these simulations, we used a learning rate of $\delta = 1$, since this effect is less robust with lower learning rates that lead to more exploration of the environment across all agents.
        {\bfseries (C)}~Policy learned by a Q-learning (top) or a SARSA (bottom) agent with $\epsilon = 0.2$.
        Colours indicate the maximum value of any action in a state from blue (-100) to yellow (+50), and arrows indicate which action has the highest value.
        The Q-learning agent learns to move right above the cliff, because this is the optimal thing to do under the assumption that subsequent actions are also optimal.
        This is because it is an `off-policy' algorithm that does not take into account the actual policy of the agent.
        In contrast, the SARSA agent learns to move a `safe distance' away from the cliff, since it is an `on-policy' algorithm that takes into account the finite probability of the agent choosing to move off the cliff from upcoming states.
        Q-learning agents are also frequently trained using a stochastic $\epsilon$-greedy policy and then evaluated with the greedy policy corresponding to $\epsilon = 0$, or they can be trained while `annealing' $\epsilon$ from some finite value to $0$ over many episodes.
        }
    \vspace*{-1.0em}
\end{figure*}



In some cases, we may not know the transition function or it could be expensive to simulate.
Additionally, research suggests that dopamine activity reflects not just state values but also \emph{action} values \citep{roesch2007dopamine,morris2006midbrain}.
This suggests an alternative algorithm for biological learning, where animals learn \emph{state-action values} in what is known as `Q-learning'.
Such Q-learning models have been commonly used to explain animal behaviour and neural activity \citep{niv2009reinforcement, mattar2018prioritized}

To understand this algorithm, we first need to define the state-action values under policy $\pi$, $Q_\pi(s, a)$:
\begin{equation}
    \label{eq:Q-values}
    Q_\pi(s,a) =  \mathbb{E}_{\tau \sim p_\pi(\tau)} \left [ \sum_{t'=t} \gamma^{t' - t} r_{t'} | s_{t} = s, a_{t} = a \right ].
\end{equation}
Given $Q_\pi$, it is simple to perform action selection by taking the action with the highest expected reward in a given state,
\begin{equation}
    a^*(s) = \text{argmax}_{a} Q_\pi(s, a).
\end{equation}
To learn the $Q$-values necessary for such action selection, we start by expanding \Cref{eq:Q-values},
\begin{equation}
    \label{eq:Q-expanded}
    Q_\pi(s,a) = r(s, a) + \gamma \sum_{s'} p(s_{t+1} = s'| s_t=s, a_t=a) \sum_{a'} \pi(a' | s') Q_\pi(s', a').
\end{equation}
For the greedy policy $\pi^g(a|s) := I_{a = a^*(s)}$, this gives rise to a self-consistency expression for the state-action values:
\begin{equation}
    \label{eq:Q-optimal}
    Q_{\pi^g}(s,a) = r(s, a) + \gamma \mathbb{E}_{s' \sim p(s' | s, a)} \left [ \text{max}_{a'} Q_{\pi^g}(s', a') \right ].
\end{equation}
Importantly, this self-consistency expression only holds when the Q-values have converged to the true expected rewards, and the associated greedy policy is therefore optimal \citep{sutton2018reinforcement}.
We can now use \Cref{eq:Q-optimal} as an objective by defining
\begin{equation}
    \mathcal{L} \propto \left (  Q(s,a) - (r(s, a) + \gamma \mathbb{E}_{s' \sim p(s' | s, a)} \left [ \text{max}_{a'} Q(s', a') \right ] ) \right )^2,
\end{equation}
Gradient descent w.r.t $Q(s,a)$ gives us an update rule
\begin{align}
    \Delta  Q(s,a) & \propto - \frac{\partial \mathcal{L}}{\partial  Q(s,a)}\\
    &\propto - Q(s,a) + r(s, a) + \gamma \mathbb{E}_{s' \sim p(s' | s, a)} \left [ \text{max}_{a'} Q(s',a') \right ] \\
    &\approx - Q(s_t,a_t) + r_t + \gamma \text{max}_{a'} Q(s_{t+1}, a').
\end{align}
This is the so-called Q-learning update rule (\citealp{watkins1989learning}; \Cref{fig:Q}A), where we have estimated the expectation with the single sample actually seen by the agent in the last line.

Q-learning is guaranteed to converge to the optimal policy in the limit of infinitesimal learning rates and infinite sampling of the state-action space \citep{watkins1992q,sutton2018reinforcement}.
However, following the greedy policy $a^*(s) = \text{argmax}_{a} Q(s, a)$ before convergence of the Q values can lead to undersampling of the state-action space and poor performance.
It is therefore common to either use an `$\epsilon$-greedy' policy, $\pi(a|s) = \epsilon / |\mathcal{A}| + (1-\epsilon) I_{a = a^*(s)}$, or a softmax-policy, $\pi(a|s) \propto \text{exp}(\beta Q(s, a))$, to collect the experience used to update the Q values (\Cref{fig:Q}B).
Such exploration strategies and their biological correlates are discussed in more detail in \Cref{sec:scalar_rew}.

These approaches make Q-learning an `off-policy' algorithms, since the policy used in the learning update (the greedy policy) is different from the policy used for action selection (the stochastic policy).
An on-policy version of Q-learning known as `SARSA' (state-action-reward-state-action) is also commonly used, where the Q-learning update uses the Q-value corresponding to the action $a_{t+1}$ sampled at the next timestep instead of the greedy action (\Cref{fig:Q}C):
\begin{align}
    \Delta  Q(s_t,a_t) \propto - Q(s_t,a_t) + r_t + \gamma Q(s_{t+1}, a_{t+1}).
\end{align}
This will converge to the true Q values for a given policy $\pi$, similar to how the TD learning rule in \Cref{eq:TD-learning} is guaranteed to converge to the true value function for a given policy, again under assumptions of infinite sampling of the space.

In settings where animals have to choose between multiple actions with different values, studies have found evidence for midbrain dopamine neurons encoding either quantities resembling the prediction error used for Q learning \citep{roesch2007dopamine,niv2009reinforcement} or SARSA \citep{morris2006midbrain,niv2009reinforcement}.
These findings suggest that such algorithms may not just be abstract learning algorithms but instead have pluasible implementations in biological neural circuits.
However, the methods considered here and in the previous section also have severe shortcomings.
For example, the amount of data needed to learn state(-action) values in an environment assumed to be stationary can be prohibitive for animals needing to act in a rapidly changing world, where bad decisions can have fatal consequences.

